\documentclass{article}
\usepackage[utf8]{inputenc}

% Used in the explanation text
\usepackage[colorlinks]{hyperref}

% Used by the template
\usepackage{setspace}
\usepackage{changepage} % to adjust margins
\usepackage[breakable]{tcolorbox}
\usepackage{float} % for tables inside tcolorbox https://tex.stackexchange.com/a/274342
\usepackage{enumitem}


\begin{document}

\newenvironment{mcsection}[1]
    {%
        \textbf{#1}

        % Reduce margins to use the space more effectively and help fit in the recommended "one to two pages"
        % Use the bullet list format as shown in the model card paper to increase readability
        \begin{itemize}[leftmargin=*,topsep=0pt,itemsep=-1ex,partopsep=1ex,parsep=1ex,after=\vspace{\medskipamount}]
    }
    {%
        \end{itemize}
    }

% Optional: reduce margins single line to fit in "one to two pages", as recommended
\begin{adjustwidth}{-60pt}{-70pt}
\begin{singlespace}

\tcbset{colback=white!10!white}
\begin{tcolorbox}[title=\textbf{Model Card -Antoine Savage-Duguay},
    breakable, sharp corners, boxrule=0.7pt]

% Change to a smaller, but still legible font size to help fit in the recommended "one to two pages"
\small{

\begin{mcsection}{Model Details}
    \item Prédiction du revenu d'un individu par classe ($> \ or \ \leq 50000 $) selon certaines caractéristique 
    \item Model développé par Antoine Savage-Duguay, étudiant à l'Université de Montréal, en Mars 2021.
    \item Régression logistique
\end{mcsection}

\begin{mcsection}{Intended Use}
    \item Le modèle à été créée à des fins académiques, et vise à explorer l'efficacité d'un algorithme d'apprentissage machine, ainsi que les potentiels enjeux pouvant émaner de l'implantation de cet algorithme.
\end{mcsection}

\begin{mcsection}{Factors}
    \item Les \textbf{facteurs pertinents} sont intrinsèquement dans la base de données, celle-ci étant déjà séparé en groupe d'individus soit : l'âge, le niveau d'éducation, le statut matrimonial, la relation conjugale actuelle, le sexe et la race. 
    \item Seulement les variables ayant le plus d'impacts sur le modèle ont été retenu, pour un meilleur "fit". Les \textbf{facteurs d'évaluation} seront donc:le niveau d'éducation, le statut matrimonial, le sexe et la race. Des facteurs d'évaluation croisé (\textbf{i.e.} sexe-race) seront également évalués.
    
    \item Il est important de souligner que \textbf{l'environnement} dans lequel les données d'entrainement du modèle date de 1994 et ont été récoltées seulement aux États-Unis.
\end{mcsection}

\begin{mcsection}{Metrics}
    \item Metrics 1....
\end{mcsection}

\begin{mcsection}{Training Data}
    \item Training data 1...
\end{mcsection}

\pagebreak

\begin{mcsection}{Evaluation Data}
    \item Evaluation data 1....
\end{mcsection}

\begin{mcsection}{Ethical Considerations}
    \item Ethical considerations 1....
\end{mcsection}

\begin{mcsection}{Caveats and Recommendations}
\item
\includegraphics[width=1\textwidth,]{histogram.png}\\[0cm]
\end{mcsection}

\textbf{Quantitative Analyses}

% Sample table inside tcolorbox
\begin{table}[H]
\centering
\small{
\begin{tabular}{lr}
Measurement 1       & 0.751  \\
Measurement 2       & 0.762 \\
Measurement 3       & 0.773 \\
Measurement 4       & 0.784 \\
Measurement average & 0.768  \\ \hline
\textbf{Model measurement}  & \textbf{0.791} \\ \hline
\end{tabular} } \\
\caption[Short caption used in list of tables.]{\small{Longer caption to explain what the measurements are.}}
\end{table}

} % end font size change
\end{tcolorbox}
\end{singlespace}
\end{adjustwidth}

\end{document}
